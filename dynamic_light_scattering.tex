\documentclass[a4paper,10pt]{article}
\usepackage{graphicx}
\usepackage{wrapfig}
\usepackage{floatrow}
\usepackage{multirow}
\usepackage[left=2cm,right=2cm,
top=2cm,bottom=2cm,bindingoffset=0cm]{geometry}
\usepackage[normalem]{ulem}  % для зачекивания текста

\usepackage{hyperref}
\usepackage[rgb]{xcolor}
\hypersetup{ % Гиперссылки
colorlinks=true, % false: ссылки в рамках
urlcolor=blue % на URL
}
\floatsetup[figure]{
captionskip=0.1cm,%
}
\usepackage{indentfirst} % Красная строка после заголовка
\setlength\parindent{1.25cm}
\usepackage[T2A]{fontenc}
\usepackage[utf8]{inputenc}
\usepackage[english,russian]{babel}
\usepackage{amsmath,amsfonts,amssymb,amsthm,mathtools}
\usepackage{wasysym}
\usepackage{floatflt}
  \usepackage{ textcomp }
  %\usepackage{floatflt}
  \usepackage{indentfirst}%красная сторка
  \usepackage[DIV=15]{typearea}
  \usepackage{fancyhdr}
  \newcommand{\bbR}{\mathbb R}%теперь вместо длинной команды \mathbb R (множество вещественных чисел) можно писать короткую запись \bbR. Вместо \bbR вы можете вписать любую строчку букв, которая начинается с '\'.
  \newcommand{\eps}{\varepsilon}
  \newcommand{\bbN}{\mathbb N}
  \newcommand{\dif}{\mathrm{d}}
  \newtheorem{Def}{Определение}
  \pagestyle{fancy}
  \setcounter{page}{2}
  \makeatletter % сделать "@" "буквой", а не "спецсимволом" - можно использовать "служебные" команды, содержащие @ в названии
  \fancyhead[L]{\footnotesize Лабораторная работа ФХМИ}%Это будет написано вверху страницы слева
  \fancyhead[R]{\footnotesize  ДРС}
  \fancyfoot[R]{}
  \setcounter{page}{0}
  \fancyfoot[C]{\thepage}
  
  \makeatother
% \def\dd#1#2{\frac{\partial#1}{\partial#2}}
  \author{Пискунова Ольга, Делинкевич Мария, Б06-205}
\addto\captionsrussian{\def\chaptername{Глава}}
\addto\captionsrussian{\def\contentsname{Содержание}}
\addto\captionsrussian{\def\listfigurename{Список рисунков}}
\addto\captionsrussian{\def\listtablename{Список таблиц}}
\addto\captionsrussian{\def\abstractname{Аннотация}}
\addto\captionsrussian{\def\refname{Список использованных источников}}
\addto\captionsrussian{\def\indexname{Предметный указатель}}
\addto\captionsrussian{\def\figurename{Рисунок}}
\addto\captionsrussian{\def\tablename{Таблица}}
\addto\captionsrussian{\def\partname{Часть}}
\addto\captionsrussian{\def\appendixname{Приложение}}
\usepackage[labelsep=endash]{caption}
\usepackage{mathrsfs}


\title{Методы статистического и динамического рассеяния света для исследования наночастиц и макромолекул в растворах}
\author{Пискунова Ольга, Делинкевич Мария, Б06-205}
\date{\today}

\begin{document}

\maketitle
\thispagestyle{empty}

\setcounter{page}{2}

% \section{Аннотация}
% Одна из важнейших характеристик дисперсных систем – это характерный линейный размер макро- и наночастиц. Существует множество различных методов измерения данного параметра, соответствующие различным диапазонам размеров частиц, на основе: центробежного осаждения частиц (от 0.5–1 нм до 5–6 мкм), электронной (от 2 нм до 1 мкм) и атомно-силовой (от 0.1 нм до 1 мкм) микроскопии. Данная работа посвящена двум методам исследования вещества: статистическому рассеянию света (СРС) и одной разновидности динамического рассеяния света (ДРС) - фотонной корреляционной спектроскопии. В диапазоне частиц от 1 нм до 6 мкм \textbf{метод динамического рассеяния} лазерного излучения определен как первичный метод измерения линейного размера частиц в жидких средах. Этот метод основан на измерении флуктуаций интенсивности рассеянного света, вызванных броуновским движением исследуемых частиц. \textbf{Метод статического светорассеяния} – один из основных количественных методов исследования полимеров в растворе. С его помощью определяют молекулярную массу и радиус инерции макромолекул. 

\section{Цель работы}
Оценить размер частиц в бразцах Au и S с помощью методов статического (СРС) и динамического рассеяния света (ДРС).
\subsection{Задачи}
• построить зав-сть обратного времени корреляции от квадрата волнового вектора рассеяния;

• построить индикатрису рассеяния;

• проанализировать автокорелляционную функцию;

• оценить гидродинамический радиус частиц
\subsection{Оборудование}
спектрометр Photocor Complex, программа DynaLS, коллоидный раствор золота Au, раствор серы S в ацетоне.
\section{Теория}
\subsection{Введение}
\textbf{Рассеяние} – это процесс, в котором молекула или частица заимствует у распространяющейся в среде электромагнитной волны некоторую долю энергии и впоследствии излучает эту энергию в окружающее пространство. Его можно представить в виде схемы из двух частей – возбуждение и переизлучение.
В результате могут происходить изменения характеристик потока излучения: пространственного распределения интенсивности, частотного спектра, поляризации. Помимо переизлучения, часть энергии падающей электромагнитной волны может быть преобразована в другие виды энергии (например, в тепло) – происходит поглощение.

Различают два типа рассеяния.
Неупругое рассеяние, приводящее к появлению в рассеянном свете линий $\omega0 \pm \Omega$, смещённых по частоте относительно возбуждающего излучения $\omega$. \textbf{Упругое} – рассеяние света, сопровождающееся перераспределением энергии
между излучением и веществом, происходящее без существенного изменения частоты. 
Диапазоны:\newline
1. \textit{Рэлея} ($d < \lambda / 15 $ ) - все элементарные диполи рассеивающей частицы излучают в одной фазе;\newline
2. \textit{Ми} ( $\lambda$ / 15 >= d <= $\lambda $) - переизлучение первичной
волны элементарными диполями;\newline
3. \textit{Фраунгофера} (d $> \lambda$) - преимущественно происходит процесс дифракции.
Методы ДРС и СРС основываются на предположении об упругом рассеянии света.
% В теории Рэлея предполагается, что электромагнитное излучение, проходя через среду,
% взаимодействует с ней, индуцируя появление диполей, являющихся источником вторичного излучения, распространяющегося во всех направлениях, кроме своей оси, с той же длиной волны, что и падающий свет.

% Теория Ми применяется, когда характерные размеры рассеивающих центров соизмеримы со
% световой длиной волны. В этом случае интенсивность рассеяния зависит как от концентрации
% раствора, так и от угла рассеяния.

% Если же размер частицы превышает длину волны падающего
% света, то происходит преимущественно процесс дифракции(Фраунгофера). Информация
% о размере частицы заключается в величине малого угла дифракционного расхождения.
\subsection{ Метод статистическкого рассеяния света}
В экспериментах по СРС регистрируется усреднённая по времени интенсивность рассеянного образцом света. Анализ угловых зависимостей и зависимостей от концентрации
интенсивности рассеянного света позволяет получить информацию о размерах и некоторых (влияющих на масштаб флуктуаций оптических свойств) термодинамических свойствах рассеивающих центров.
Во многих практических приложениях используются макромолекулы и частицы с размерами в диапазоне от нанометров до микрон. Под размером в методах СРС подразумевается радиус инерции - Rg - величина, характеризующая распределение элементарных
излучателей в частице или макромолекуле:
\[ R^2_{g} = \frac{ \sum \limits_{i} m_{i} r^2_{i}}{\sum\limits_{i} m_{i}} \]


Суммарная интенсивность рассеянного света I зависит от длины световой волны $\lambda$
(в вакууме), интенсивности падающего света $I_{0}$, рассеивающего объёма $\omega$, расстояния
от рассеивающего объёма до приёмника x, поляризуемости частицы $\alpha$, концентрации
рассеивающих частиц $n_{0}$ и угла рассеяния $\theta$:
\[I =\frac{16π^4}{\lambda^4x^2 }\alpha^2n_{0}\Omega I_{0} P(\Theta)\]

где P($\Theta$) - коэффициент формы.
Поляризуемость излучающей частицы, пропорциональна объёму частицы , если образующие частицу элементарные диполи излучают в одной фазе (это условие выполняется
для достаточно малых частиц) $\alpha \sim d^3$
. Из приведённой выше формулы следует, что
интенсивность пропорциональна шестой степени размера частицы: $ I \sim \alpha^2 \sim d^6$
.

\textbf{Индикатриса рассеяния} - диаграмма направленности излучения, графически отображающая зависимость интенсивности рассеянного света от угла рассеяния.
Вид угловой зависимости рассеяния $P(\Theta)$ определяется размерами и формой рассеивающих частиц.

В зависимости от размеров рассеивающих частиц, показателей преломления и направления наблюдения интерференция волн может усиливать или ослаблять интенсивность рассеянного света. В случае рэлеевского рассеяния все диполи внутри частицы излучают в одной фазе, поэтому интенсивность рассеянного излучения для всех $\theta$ совпадает с
максимально возможной (при заданном количестве элементарных диполей).
Для более крупных частиц удобно провести нормировку на максимальную интенсивность рассеянного излучения, рассчитываемую по теории Рэлея. Эффекты интерференции описываются коэффициентом формы $P(\Theta)$, введенным ранее:
\[I(\Theta) = P(\Theta) · I(0) = P(\Theta) · I_{Rayleight} \]
Коэффициент формы всегда меньше или равен 1.
\subsection{Метод динамического рассеяния света}
Методы динамического светорассеяния
позволяют определить время жизни флуктуации. Одной из разновидностей ДРС является метод фотонной корреляционной спектроскопии.

В данном методе изучается корреляция (во времени) количества фотонов,
рассеянных образцом в заданном направлении. В качестве источника излучения
используется лазер. Электромагнитные волны, рассеянные соседними частицами, интерферируют друг с другом. Возникающие при этом временные флуктуации интенсивности рассеянного света формируют на фотодетекторе сигнал I(t), характерные времена изменения которого обусловлены броуновским движением рассеивающих частиц. 


Прибор, называемым коррелятором, который строит автокорреляционную функцию сигнала. втокорреляционная функция показывает корреляцию значений сигнала (в
данном случае – интенсивности рассеянного света) измеренных через промежу-
ток времени $\tau$.
\begin{equation}
    g_2(\tau) = <I(t)I(t+\tau)>
\end{equation}

АКФ электрического поля, называемая корреляционной функцией первого
порядка (в отличие от корреляционной функции второго порядка для
интенсивности поля), вводится аналогично:
\begin{equation}
    g_1(\tau) = <E^*(t)E(t+\tau)>
\end{equation}

Электрическое поле как падающей, так и рассеянной волн линейно поляризовано, поэтому рассмотрим только z-компоненту поля. Электрическое поле световой волны, рассеянной на флуктуациях показателя преломления среды в направлении вектора $k'$, можно представить в виде $E(t) = \delta E(t)*e^{-i\omega_0 t)}$.
% Медленно меняющаяся со временем амплитуда поля $ \delta E(t)$ пропорциональна флуктуации концентрации рассеивающих частиц с волновым вектором q, ответственным за рассеяние согласно условию
% Брэгга:
% \begin{equation}
%     \delta C_{\Vec{q}}(\Vec{r}, t) = \delta A \cdot sin(\Vec{q} \cdot \Vec{r})
% \end{equation}
% \begin{equation}
%     \delta E(t)  = A \cdot \delta C_{\Vec{q}}(\Vec{r}, t).
% \end{equation}

Здесь подразумевается, что флуктуации представлены в виде пространственного Фурье-разложения. В соответствии с гипотезой Онзагера, релаксация микроскопических флуктуаций концентрации к равновесному состоянию может быть описана уравнением диффузии:
\begin{equation}
    \frac{\partial C}{\partial t} = D \Delta C.
\end{equation}
Согласно этому уравнению флуктуации концентрации экспоненциально затухают с течением времени.
Причём величина, обратная времени жизни такой флуктуации, равна:
\begin{equation}
    1/t_{c}= D \cdot q^2.
\end{equation}

В этом случае автокорреляционная функция электрического поля рассеянного излучения затухает также по экспоненциальному закону с тем же характерным временем.
Таким образом, по результатам аппроксимации автокорреляционной функции интенсивности рассеянного света можно определить коэффициент диффузии частиц. Далее, исходя из него, размер частиц может быть найден из формулы Стокса–Эйнштейна:

\begin{equation}
    D = \frac{k_{b} T}{6\pi R \eta}.
\end{equation}

\section{Прибор}
В данной работе используется Photocor Complex для измерения параметров диспергированных в жидкости наноразмерных частиц. 


На жестком основании смонтированы прецизионный гониометр и оптическая скамья, на которой размещены диодный лазер Photocor-DL ($\lambda_{0}=$
658.6 нм) и фокусирующий узел. Термостат и адаптер кювет установлены на
гониометре коаксиально с его осью. На поворотной консоли гониометра располагается фотоприемный блок, в состав которого входит приемная оптическая
система со сменной диафрагмой выбора апертуры, малошумящий фотоумножитель, работающий в режиме счета фотонов, быстрый усилитель-дискриминатор
со сквозным по постоянному току трактом и специальный высоковольтный источник питания ФЭУ без паразитных корреляций. 
\begin{figure}[H]
    \centering
    \includegraphics[width=0.5\linewidth]{гыефт.png}
    \caption{Принципиальная схема прибора.}
    \label{fig:enter-label}
\end{figure}

\begin{figure}[H]
    \centering
    \includegraphics[width=0.5\linewidth]{зрщещсщк.png}
    \caption{Схема прибора Photocor Complex.}
    \label{fig:enter-label}
\end{figure}

\section{Ход работы}
В данной работе исследовались два образца: золь золота и сера в ацетоне. Кювета с исследуемым раствором помещается в кюветное отделение анализатора. Пучок света, испускаемый лазером, рассеивается на дисперсных частицах, находящихся в растворе. Рассеянный свет принимается системой счета фотонов, сигнал с выхода которой подается на вход коррелятора. Коррелятор накапливает корреляционную функцию
флуктуаций интенсивности рассеянного света. По завершении выбранного времени измерения корреляционная функция передается в компьютер в виде
функции, связанной с АКФ линейной зависимостью.Фотоприемный блок фиксируется под определенным углом от 20$^{\circ}$ до 140$^{\circ}$. Компьютер (программа обработки DynaLS, описанная ниже) рассчитывает размер частиц, обрабатывая измеренную корреляционную функцию.
\subsection{Сера}
\subsubsection{Оценка размера частиц}

\begin{figure}[H]
    \centering
    \includegraphics[width=0.7\linewidth]{дштыукф.png}
    \caption{Зависимость обратного времени когерентности от квадрата разности волновых векторов.}
    \label{fig:enter-label}
\end{figure}
По зависимости из формулы 4: 
\begin{center}
    $D = (99,7 \pm 5,6)\cdot 10^{13} \frac{\text{м}^2}{c}$ 
\end{center}
C помощью формулы 5 определим радиус частиц серы:
\begin{center}
    $R = (245\pm 25)$~нм
\end{center}

\subsubsection{Индикатриса рассеяния}
\begin{figure}[H]
    \centering
    \includegraphics[width=0.5\linewidth]{seratabe.png}
    \caption{Данные для построения индикатрисы рассеяния.}
    \label{fig:enter-label}
\end{figure}

\begin{figure}[H]
    \centering
    \includegraphics[width=.5\linewidth]{indsera.png}
    \caption{Зависимость интенсивности рассеянного света от угла рассеяния.}
    \label{fig:enter-label}
\end{figure}

Индикатриса рассеяния ассиметрична, что говорит о росте размера частиц. Похоже, применимо приближение Ми.

\subsection{Золото}

\subsubsection{Оценка размера частиц}

\begin{figure}[H]
    \centering
    \includegraphics[width=0.8\linewidth]{золотолиния.png}
    \caption{Зависимость обратного времени когерентности от квадрата разности волновых векторов.}
    \label{fig:enter-label}
\end{figure}
По зависимости из формулы 4:
\begin{center}
    $D = (144,0 \pm 7,8)\cdot 10^{13} \frac{\text{м}^2}{c}$ 
\end{center}
C помощью формулы 5 определим радиус частиц золота:
\begin{center}
    $R = (17\pm 5)$~нм
\end{center}


\subsubsection{Индикатриса рассеяния}

\begin{figure}
    \centering
    \includegraphics[width=0.5\linewidth]{золотоданные.png}
    \caption{Данные для построения индикатрисы рассеяния.}
    \label{fig:enter-label}
\end{figure}

\begin{figure}[H]
    \centering
    \includegraphics[width=0.5\linewidth]{ind_au.png}
    \caption{Диаграмма в полярных координатах зависимости интенсивности рассеянного света от угла рассеяния.}
    \label{fig:enter-label}
\end{figure}


Интенсивность излучения равномерно распределена на всех измеренных значениях углов, следовательно, наблюдаем рэлеевский характер рассеивания на коллоидном растворе золота, тогда частицы не превышают $\lambda/15 \leq 44$ нанометров в размере. 

\subsection{Автокорреляционная функция}
 Для угла 90$^{\circ}$ для образцов серы и золота проведем анализ автокорелляционной функции, имеющей следующий вид:
\begin{equation}
   G(t) = \langle I(t)I(t-\tau) \rangle = \lim_{\Delta t\rightarrow\infty}\frac{1}{\Delta t}\int_{0}^{\Delta t} I(t)I(t-\tau)\,dt
\end{equation}


\begin{figure}[H]
    \centering
    \includegraphics[width=0.5\linewidth]{gdgdg.png}
    \caption{Автокррелляционная функция для серы и золота.}
    \label{fig:enter-label}
\end{figure}

Видно, что частицы золота меньше, т.к. экспонента начинает убывать на меньших каналах соответствующих меньшим временам. Каналы и времена сконвертированы в соответствии с руководством пользователя ($\tau = 2^{n/8} \cdot 10^{-7}$).

Приблизим кривые экспонентой:
\begin{equation*}
    g_{1}(\tau) = p\exp{\left(-\frac{\tau}{t_{c}}\right)}
\end{equation*}

\begin{figure}[H]
    \centering
    \includegraphics[width=0.5\linewidth]{zoll.png}
    \caption{Автокоррелляционная функция для золота.}
    \label{fig:enter-label}
\end{figure}

\begin{figure}[H]
    \centering
    \includegraphics[width=0.5\linewidth]{sera.png}
    \caption{Автокоррелляционная функция для серы.}
    \label{fig:enter-label}
\end{figure}
\begin{equation*}
    Au: t_{c} = 0.00023715 \text{ сек}
\end{equation*}
\begin{equation*}
    S: t_{c} = 0.003435 \text{ сек}
\end{equation*}

При $90^{\circ}: q = 0.018~\text{нм}^{-1}$. Тогда из формул 4 и 5:  $R_{Au} = (19\pm 2)$~нм,  $R_{S} = (273\pm 20)$~нм.

\section{Вывод}
В результате применения метода статического рассеяния (СРС) света получены индикатрисы рассеяния серы S и золота Au. Из диаграмма оценка размера частиц: для золота - десятки нм, для серы - сотни нм. Для оценки размера частиц был также применени метод динамического рассеяния света (ДРС). В результате для частицы золота оценочный средний размер 17 нм, серы - 245 нм. По результатам анализа автокррелляционной кривой размер частиц золота 19 нм, серы - 273 нм. Результаты совпали в пределах погрешности.


\end{document}
